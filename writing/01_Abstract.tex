In retrieved context accounts of memory search such as the Context Maintenance and Retrieval (CMR) model \citep*{polyn2009context}, representations of studied items in a free recall experiment are associated with an internal representation of context that changes slowly during the study period. These associations in turn account for organizational effects in recall sequences, such as the tendency for related items to be recalled successively. Specifications of the model tend to characterize these dynamics in terms of a simplified neural network, as building a single prototypical pattern of associations between each item and context (and vice versa) across experience. 